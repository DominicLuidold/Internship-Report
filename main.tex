\documentclass[a4paper,12pt,twoside]{scrreprt}
% Autor der Vorlage: Klaus Rheinberger, FH Vorarlberg, 2017-02-20

% Pakete:
\usepackage[utf8]{inputenc}
\usepackage[T1]{fontenc} % Silbentrennung bei Sonderzeichen
\usepackage{graphicx} % Bilder einbinden
\usepackage[ngerman]{babel} % Deutsche Sprachanpassungen
\usepackage{csquotes} % When using babel or polyglossia with biblatex, loading csquotes is recommended to ensure that quoted texts are typeset according to the rules of your main language.
\usepackage{acronym} % Abkürzungsverzeichnis
\usepackage[linktocpage=true]{hyperref} % Links -> \href{https://www.wikibooks.org}{Wikibooks home}
\usepackage[bindingoffset=8mm]{geometry} % Bindeverlust von 8mm einbeziehen
\usepackage{caption} % Abbildungslegenden
\captionsetup{format=hang, justification=raggedright}

% Literaturverweise:
\usepackage[style=authoryear,citestyle=authoryear,backend=biber]{biblatex}   
\addbibresource{Zotero-Beispiele.bib}

% Einstellungen:
\setcounter{secnumdepth}{4}
\setcounter{tocdepth}{4} % Tiefe der Gliederung im Inhaltsverzeichnis

\begin{document}

% evtl. Sperrvermerkseite
\thispagestyle{empty}
[Achtung: Verwenden Sie einen Sperrvermerk nur in sehr gut begründeten Fällen!] 

\section*{[evtl. Sperrvermerk]}   % evtl. ersetzen durch \section*{Sperrvermerk}
Auf Wunsch der Firma [FIRMA] ist die vorliegende Arbeit bis zum [DATUM] für die öffentliche Nutzung zu sperren. 

Veröffentlichung, Vervielfältigung und Einsichtnahme sind ohne ausdrückliche Genehmigung der oben genannten Firma und der/dem Verfasser/in nicht gestattet. Der Titel der Arbeit sowie das Kurzreferat/Abstract dürfen jedoch veröffentlicht werden.

\vspace{3cm}

\noindent Dornbirn, \hfill Unterschrift der Verfasserin/des Verfassers

\vspace{2cm}

\hfill Firmenstempel\hspace{2cm}


% Titelblatt:
% \newpage\mbox{}\newpage
\cleardoublepage   % force output to a right page
\thispagestyle{empty}
\begin{titlepage}
  \begin{flushright}
  \includegraphics[width=0.4\linewidth]{Logo-A3}
  \end{flushright}
  \begin{flushleft}
  \section*{[Titel der Arbeit]}
  \subsection*{[Untertitel der Arbeit]}
  \vspace{1cm}
  
  Bachelorarbeit I\\
  zur Erlangung des akademischen Grades
  \vspace{0.5cm}
  
  \textbf{Bachelor of Science in Engineering (BSc)}

  \vspace{1cm}
  Fachhochschule Vorarlberg\newline
  Informatik – Software and Information Engineering

  \vspace{0.5cm}
  
  Betreut von\newline
  Prof. (FH) Dipl. Inform. Thomas Feilhauer
  
  \vspace{0.5cm}
  
  Vorgelegt von\newline
  Dominic Luidold\newline
  Dornbirn, November 2020
  \end{flushleft}
\end{titlepage}

% Widmung:
\newpage
\section*{[Widmung]}

[Text der Widmung]

% Kurzreferat:
\newpage
\section*{Kurzreferat}

\subsection*{[Deutscher Titel Ihrer Arbeit]}

[Text des Kurzreferats]

% Abstract:
\newpage
\section*{Abstract}
\subsection*{[English Title of your thesis]}

[text of the abstract]

% evtl. Vorwort:
\newpage
\section*{[evtl. Vorwort]}

[Text des Vorworts]

% Inhaltsverzeichnis:
\cleardoublepage % force output to a right page
\tableofcontents

\clearpage
\phantomsection
\addcontentsline{toc}{chapter}{Abbildungsverzeichnis}
\listoffigures

\clearpage
\phantomsection
\addcontentsline{toc}{chapter}{Tabellenverzeichnis}
\listoftables

% Abkürzungsverzeichnis:
\clearpage
\phantomsection
\addcontentsline{toc}{chapter}{Abkürzungsverzeichnis}
\chapter*{Abkürzungsverzeichnis}
\begin{acronym}
 \acro{BLS}{Better Life System}
\end{acronym}

\chapter{Einleitung}
Diese Bachelorarbeit verfolgt das Ziel, einen Einblick in das bereits bestehende Backend-System des "Better Life System" (kurz "BLS") der Viterma Handels GmbH, basierend auf der Programmiersprache PHP und dem PHP-Framework Symfony, zu geben.\\
Das Better Life System, welches mittels eines Angular-Frontends in gängigen Webbrowsern bedient werden kann, erlaubt es Anwenderinnen mithilfe eines Handelsvertreters der Viterma GmbH, ein Badezimmer nach ihren Bedürfnissen 

Das PHP-Backend dient dabei als Schnittstelle zwischen einem Datenbankserver und einem auf Angular aufbauenden JavaScript Frontend.\\
Im Verlauf der Arbeit wird beleuchtet, wie 

Formatvorlage für den Fließtext. Formatvorlage für den Fließtext. Formatvorlage für den Fließtext. Formatvorlage für den Fließtext. Formatvorlage für den Fließtext. Formatvorlage für den Fließtext. Formatvorlage für den Fließtext. Formatvorlage für den Fließtext. Formatvorlage für den Fließtext. Formatvorlage für den Fließtext. Formatvorlage für den Fließtext.
\begin{quote}
  Formatvorlage für ein längeres direktes Zitat. Formatvorlage für ein längeres direktes Zitat. Formatvorlage für ein längeres direktes Zitat. Formatvorlage für ein längeres direktes Zitat. Formatvorlage für ein längeres direktes Zitat. Formatvorlage für ein längeres direktes Zitat….
\end{quote}

\href{https://www.wikibooks.org}{Wikibooks home}

\chapter{[Kapitel]}
Formatvorlage für den Fließtext.
Hier eine Liste.
\begin{enumerate}
 \item Verstehen
 \item Üben
 \item Können
\end{enumerate}


\section{[Unterkapitel zweite Ebene]}
Formatvorlage für den Fließtext. Die Abbildung~\ref{fig:ex} auf Seite \pageref{fig:ex} zeigt drei Entladungskurven eines biphasischen Defibrillators.
\begin{figure}[htb]
  \centering
  \includegraphics[width=10cm]{Amann_TechnAbb}
  \caption[Aufheizverhalten von PTFE]{Das Bild zeigt das Aufheizverhalten von PTFE. \\Quelle: eigene Ausarbeitung}
 \label{fig:ex}
\end{figure}


\section{[Unterkapitel zweite Ebene]}
Formatvorlage für den Fließtext.
Jetzt eine Fußnote\footnote{Dies ist eine Fußnote.}
Die quadratischen Gleichung (\ref{equ:foo}) hat wieviele Nullstellen?
\begin{equation}
 \label{equ:foo}
 x^2-2x+5=0.
\end{equation}
Zwei von Einsteins berühmtesten Formeln lauten:
\begin{eqnarray*}
  E &= mc^2                                  \\
  m &= \frac{m_0}{\sqrt{1-\frac{v^2}{c^2}}}
\end{eqnarray*}


\subsection{[Unterkapitel dritte Ebene]}
Formatvorlage für den Fließtext. Hier die einfache Tabelle \ref{tab:sp}

\begin{table}[htb]
  \centering
  \begin{tabular}{ | l | l |c|}
    \hline
    Datum      & Thema           & Raum \\
    \hline\hline
    Montag     & Graphentheorie  & U1   \\
    \hline
    Donnerstag & Algebra         & MZB23\\
    \hline
  \end{tabular}
  \caption[Stundenplan]{Stundenplan des Jahres 2030.\\Quelle: eigene Ausarbeitung}
  \label{tab:sp}
\end{table}

\subsubsection{[Unterkapitel vierte Ebene]}
Formatvorlage für den Fließtext.



\section{[Unterkapitel zweite Ebene]}

Verweise: zu einem Buch mit Details \cite[vgl.][Kapitel 2]{bathe_finite-elemente-methoden_1990} oder ohne Details \cite{bathe_finite-elemente-methoden_1990}, ein Buchteil \cite{areger_problem-based_2007}, eine Dissertation \cite{sporn_interaktives_2000}, ein Dokument \cite{industriellenvereinigung_beste_2014}, ein Enzyklopädieartikel \cite{brockhaus_kreativitat_1872}, ein Film \cite{de_wilde_through_2008}, ein Konferenz-Paper \cite{weber_podcasts._2006}, ein Magazin-Artikel \cite{autornachname1_magazinartikeltitel_1995}, ein Pordcast \cite{paulus_horen_????}, eine Tonaufnahme \cite{horowitz_horowitz_2003}, eine Videoaufnahme \cite{fhvlearningsupport_was_2008}, ein Vortrag \cite{kohls_literaturverwaltung_2008}, eine Website \cite{wedekind_von_2008}, ein Zeitschriftenartikel \cite{hofer_wir_2008} und ein Zeitungsartikel \cite{schenkel_tsunami_2012}.


\chapter{[Kapitel]}

\section{[Unterkapitel zweite Ebene]}
Formatvorlage für den Fließtext.

\subsection{[Unterkapitel dritte Ebene]}
Formatvorlage für den Fließtext.

\subsubsection{[Unterkapitel vierte Ebene]}
Formatvorlage für den Fließtext.


% Literaturverzeichnis:
\clearpage
\phantomsection
\addcontentsline{toc}{chapter}{Literaturverzeichnis}
\printbibliography

\chapter*{[evtl. Anhang]}  % evtl. ersetzen mit \chapter*{Anhang}
\addcontentsline{toc}{chapter}{[evtl. Anhang]}   % evtl. ersetzen mit \addcontentsline{toc}{chapter}{Anhang}
Formatvorlage für den Fließtext.


\chapter*{Eidesstattliche Erklärung}
\addcontentsline{toc}{chapter}{Eidesstattliche Erklärung}
Ich erkläre hiermit an Eides statt, dass ich die vorliegende Masterarbeit selbstständig und ohne Benutzung anderer als der angegebenen Hilfsmittel angefertigt habe. Die aus fremden Quellen direkt oder indirekt übernommenen Stellen sind als solche kenntlich gemacht. Die Arbeit wurde bisher weder in gleicher noch in ähnlicher Form einer anderen Prüfungsbehörde vorgelegt und auch noch nicht veröffentlicht.

\vspace{3cm}
\noindent
Dornbirn, am [Tag. Monat Jahr anführen]\hfill [Vor- und Nachname Verfasser/in]

\end{document}
