\documentclass[a4paper,12pt,twoside]{scrreprt}
% Autor der Vorlage: Klaus Rheinberger, FH Vorarlberg, 2017-02-20

% Pakete:
\usepackage[utf8]{inputenc}
\usepackage[T1]{fontenc} % Silbentrennung bei Sonderzeichen
\usepackage{graphicx} % Bilder einbinden
\usepackage[ngerman]{babel} % Deutsche Sprachanpassungen
\usepackage{csquotes} % When using babel or polyglossia with biblatex, loading csquotes is recommended to ensure that quoted texts are typeset according to the rules of your main language.
\usepackage{acronym} % Abkürzungsverzeichnis
\usepackage[linktocpage=true]{hyperref} % Links -> \href{https://www.wikibooks.org}{Wikibooks home}
\usepackage[bindingoffset=8mm]{geometry} % Bindeverlust von 8mm einbeziehen
\usepackage{caption} % Abbildungslegenden
\usepackage{xcolor} % Farbige Hervorhebungen
\usepackage{setspace} % Zeilenabstand
\captionsetup{format=hang, justification=raggedright}

% Literaturverweise:
\usepackage[style=authoryear,citestyle=authoryear,backend=biber]{biblatex}   
\addbibresource{Zotero.bib}

% Einstellungen:
\setcounter{secnumdepth}{4}
\setcounter{tocdepth}{4} % Tiefe der Gliederung im Inhaltsverzeichnis

\begin{document}
\onehalfspacing % Zeilenabstand 1,5

% evtl. Sperrvermerkseite
\thispagestyle{empty}
[Achtung: Verwenden Sie einen Sperrvermerk nur in sehr gut begründeten Fällen!] 

\section*{[evtl. Sperrvermerk]}   % evtl. ersetzen durch \section*{Sperrvermerk}
Auf Wunsch der Firma [FIRMA] ist die vorliegende Arbeit bis zum [DATUM] für die öffentliche Nutzung zu sperren. 

Veröffentlichung, Vervielfältigung und Einsichtnahme sind ohne ausdrückliche Genehmigung der oben genannten Firma und der/dem Verfasser/in nicht gestattet. Der Titel der Arbeit sowie das Kurzreferat/Abstract dürfen jedoch veröffentlicht werden.

\vspace{3cm}

\noindent Dornbirn, \hfill Unterschrift der Verfasserin/des Verfassers

\vspace{2cm}

\hfill Firmenstempel\hspace{2cm}


% Titelblatt:
% \newpage\mbox{}\newpage
\cleardoublepage   % force output to a right page
\thispagestyle{empty}
\begin{titlepage}
  \begin{flushright}
  \includegraphics[width=0.4\linewidth]{Logo_FHV.jpg}
  \end{flushright}
  \begin{flushleft}
  \section*{[Titel der Arbeit]}
  \subsection*{[Untertitel der Arbeit]}
  \vspace{1cm}
  
  Bachelorarbeit I\\
  zur Erlangung des akademischen Grades
  \vspace{0.5cm}
  
  \textbf{Bachelor of Science in Engineering (BSc)}

  \vspace{1cm}
  Fachhochschule Vorarlberg\newline
  Informatik – Software and Information Engineering

  \vspace{0.5cm}
  
  Betreut von\newline
  Prof. (FH) Dipl. Inform. Thomas Feilhauer
  
  \vspace{0.5cm}
  
  Vorgelegt von\newline
  Dominic Luidold\newline
  Dornbirn, November 2020
  \end{flushleft}
\end{titlepage}

% Widmung:
\newpage
\section*{[Widmung]}

[Text der Widmung]

% Kurzreferat:
\newpage
\section*{Kurzreferat}

\subsection*{[Deutscher Titel Ihrer Arbeit]}

[Text des Kurzreferats]

% Abstract:
\newpage
\section*{Abstract}
\subsection*{[English Title of your thesis]}

[text of the abstract]

% Vorwort:
\newpage
\section*{Vorwort}

Der Verfasser der vorliegenden Arbeit bekennt sich zu einer geschlechtergerechten Sprachverwendung.\\
Um die Lesbarkeit zu gewährleisten und zugunsten der Textökonomie werden die verwendeten Personen bzw. Personengruppen fix männlich oder weiblich zugeordnet. Zum Beispiel wird immer „die Entwicklerin“ und „der Benutzer“ verwendet. Es wurde besonders darauf geachtet, stereotype Rollenbeschreibungen zu vermeiden. Die insgesamt eventuell dadurch hervorgerufene Irritation bei den Lesenden ist gewünscht und soll dazu beitragen, eine Bewusstheit für die bestehende, Frauen diskriminierende Sprachgewohnheit (generelle Verwendung der männlichen Begriffe für beide Geschlechter) zu wecken bzw. zu stärken. 

% Inhaltsverzeichnis:
\cleardoublepage % force output to a right page
\tableofcontents

\clearpage
\phantomsection
\addcontentsline{toc}{chapter}{Abbildungsverzeichnis}
\listoffigures

\clearpage
\phantomsection
\addcontentsline{toc}{chapter}{Tabellenverzeichnis}
\listoftables

% Abkürzungsverzeichnis:
\clearpage
\phantomsection
\addcontentsline{toc}{chapter}{Abkürzungsverzeichnis}
\chapter*{Abkürzungsverzeichnis}
\begin{acronym}
 \acro{BLS}{Better Life System}
 \acro{CQRS}{Command-Query-Responsibility-Segregation}
\end{acronym}

\chapter{Einleitung}
Diese Bachelorarbeit verfolgt das Ziel, einen Einblick in die Implementierung und Erweiterung des bereits bestehenden Backend-Systems des {\enquote{Better Life System}} (kurz \enquote{BLS}) der Viterma Handels GmbH mit dem sogenannten \enquote{CQRS}-Pattern zu geben.\\
Das Better Life System - welches von Fusonic GmbH entwickelt wird und mittels einer Weboberfläche mit gängigen Webbrowsern bedient werden kann - ermöglicht es Endkunden mithilfe einer Handelsvertreterin der Firma Viterma (beziehungsweise mit einer Vertreterin eines ihrer Franchise-Partner) ein Badezimmer anhand der jeweiligen Bedürfnisse auszusuchen und zu konfigurieren und sich schlussendlich ein entsprechendes Angebot dafür erstellen zu lassen.

\section{Arbeitgeber Fusonic GmbH}
Die Fusonic GmbH hat ihren Firmensitz in Götzis, Vorarlberg, und besteht aus einem Team von aktuell mehr als 25 Mitarbeitern, Softwareentwicklerinnen und Projektleiterinnen. Diese sind aufgeteilt in diverse kleinere Teams, die intern unter anderem \enquote{Duck-Team} beziehungsweise \enquote{Parrot-Team} genannt werden. Die Teams arbeiten dabei an jeweils eigenständigen Projekten und setzen diverse Technologien ein. Zu den verwendeten Technologien gehören unter anderem C\# mit .NET, PHP mit Symfony sowie JavaScript/TypeScript mit Angular. \parencite[vgl.]["Übersicht aller Technologien"]{fusonic_gmbh_web_nodate} Während es  regelmäßigen Austausch unter den Teams gibt, besteht jedes aus Frontend- sowie Backend-Entwicklern, da die meisten Projekte aufgrund der zugrundeliegenden Anwendungsfälle aus einem Backend sowie Web-Frontend bestehen.

% Literaturverzeichnis:
\clearpage
\phantomsection
\addcontentsline{toc}{chapter}{Literaturverzeichnis}
\printbibliography

\chapter*{[evtl. Anhang]}  % evtl. ersetzen mit \chapter*{Anhang}
\addcontentsline{toc}{chapter}{[evtl. Anhang]}   % evtl. ersetzen mit \addcontentsline{toc}{chapter}{Anhang}
Formatvorlage für den Fließtext.


\chapter*{Eidesstattliche Erklärung}
\addcontentsline{toc}{chapter}{Eidesstattliche Erklärung}
Ich erkläre hiermit an Eides statt, dass ich die vorliegende Masterarbeit selbstständig und ohne Benutzung anderer als der angegebenen Hilfsmittel angefertigt habe. Die aus fremden Quellen direkt oder indirekt übernommenen Stellen sind als solche kenntlich gemacht. Die Arbeit wurde bisher weder in gleicher noch in ähnlicher Form einer anderen Prüfungsbehörde vorgelegt und auch noch nicht veröffentlicht.

\vspace{3cm}
\noindent
Dornbirn, am [Tag. Monat Jahr anführen]\hfill [Vor- und Nachname Verfasser/in]

\end{document}
